\documentclass[12pt, a4paper]{article}
\usepackage[a4paper, includeheadfoot, mag=1000, left=2cm, right=1.5cm, top=1.5cm, bottom=1.5cm, headsep=0.8cm, footskip=0.8cm]{geometry}
% Fonts
\usepackage{fontspec, unicode-math}
\setmainfont[Ligatures=TeX]{CMU Serif}
\setmonofont{CMU Typewriter Text}
\usepackage[english, russian]{babel}
% Indent first paragraph
\usepackage{indentfirst}
\setlength{\parskip}{5pt}
% Page headings
\usepackage{fancyhdr}
% Diagrams
\usepackage{graphicx}
% Code listing
\usepackage{listings}
\pagestyle{fancy}
\renewcommand{\headrulewidth}{0pt}
\setlength{\headheight}{16pt}
\newfontfamily\namefont[Scale=1.2]{Gloria Hallelujah}
\fancyhead{}
% Use case template

\begin{document}

% Title page
\begin{titlepage}
\begin{center}

\textsc{ITMO University}
\vfill
\textbf{Labwork №4\\[4mm]
System Programming Languages}\\[16mm]
\begin{flushright}
~\\[2mm]Sarzhevskiy Ivan
~\\[2mm]Group P3202
\end{flushright}
\vfill
Saint-Petersburg\\[2mm]
2018

\end{center}
\end{titlepage}


\section*{Assignment}
\subsection*{Scalar product}
The solution should consist of:
\begin{itemize}
    \item Two global arrays of the same size.
    \item The function to compute the scalar product of two given arrays.
    \item A main function that calls the product computations and outputs it's results.
\end{itemize}

\subsection*{Prime Number Checker}
You have to write a function to test the number for primarity. The interesting thing is that the number will be 
of the type unsigned long and that it will be read from stdin.
\begin{itemize}
    \item You have to write a function int is\_prime(unsigned long n), which checks whether n is a prime number or not. If it is the case, the function will return 1; otherwise 0.
    \item The main function will read an unsigned long number and call is\_prime function on it. Then, depending on its result, it will output either yes or no.
    \item Use scanf function with the format specifier \%lu
\end{itemize}

\section*{Code listing}
\subsection*{scalar.c}
\lstinputlisting[language=c]{../scalar.c}

\subsection*{main\_scalar.c}
\lstinputlisting[language=c]{../main_scalar.c}

\subsection*{is\_prime.c}
\lstinputlisting[language=c]{../is_prime.c}

\subsection*{main\_prime\_test.c}
\lstinputlisting[language=c]{../main_prime_test.c}

\section*{Conclusion}
While compliting the assignment I met the basics of C language and wrote scalar product function and prime number checking function.
\end{document}
